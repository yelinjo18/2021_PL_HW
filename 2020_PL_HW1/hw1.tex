% !TEX TS-program = pdflatex
% !TEX encoding = UTF-8 Unicode

\documentclass{article}

\usepackage{amsmath,amssymb}
\usepackage{kotex}
\usepackage{listings}

\begin{document}
	\title{프로그래밍 언어 HW1}
	\author{B811181 조예린}
	\maketitle
	
	\section{과제 1}
	리스트를 입력 받아 Binary Search로 숫자 찾는 문제
	\subsection{리스트 입력 받기}
	map(), split() 이라는 함수를 통해 input()으로 입력 받고 int형으로 변환해주었다.
	\begin{lstlisting}[language=Python]
	my_list = list(map(int,input().split()))
	\end{lstlisting}
	\subsection{BS 함수}
	리스트와 찾을 숫자를 인자로 받는 BS 함수를 정의하였다.
	\begin{lstlisting}[language=Python]
def BS(list, key):
	low=0
	high=len(list)-1
	# Search key!
	while low<=high:
		m = int((low + high) / 2)
		if list[m]==key:
			break
		elif list[m]<key:
			low=m+1
		elif list[m]>key:
			high=m-1

	if low>high:
		print("None")
	else:
		print(m+1)
	\end{lstlisting}
	
	
	\section{과제 2}
	리스트를 Quick Sort로 정렬
	\subsection{리스트 입력 받기}
	1.1과 동일
	\subsection{QuickSort 함수}
	Recursion으로 QuickSort 함수 구현
	인자로는 리스트의 처음과 끝 index를 받았다.
	\begin{lstlisting}[language=Python]
# Quick Sort
def QuickSort(first_i, last_i):
	if(first_i<last_i):
		# Partition
		pivot=list[first_i]
		i=first_i+1
		j=last_i
	
		while i<=j:
			while i<=last_i and list[i]<pivot:
				i+=1
			while j>first_i and list[j]>pivot:
				j-=1
	
			if i>j:
				# Exchange pivot & list[j]
				list[first_i]=list[j]
				list[j]=pivot
			else:
				# Exchange list[i] & list[j]
				tmp=list[i]
				list[i]=list[j]
				list[j]=tmp
				
		# Recursion
		QuickSort(first_i, j-1)
		QuickSort(j+1, last_i)
	\end{lstlisting}


	\section{과제 3}
		리스트를 Merge Sort로 정렬
	\subsection{리스트 입력 받기}
	1.1과 동일
	\subsection{MergeSort 함수}
	Recursion으로 MergeSort 함수 구현
	인자로 리스트를 받고 리스트를 반환하였다.
	\begin{lstlisting}[language=Python]
#Merge Sort
def MergeSort(list):
	if len(list)<=1:
		return list
	else:
		m_i=int(len(list)/2)

		# Recursion
		pre=MergeSort(list[:m_i])
		post=MergeSort(list[m_i:])

		return Merge(pre, post)
	\end{lstlisting}
	\subsection{Merge 함수}
	두 리스트를 받아 Merge한 후 하나의 리스트로 반환하였다.
	\begin{lstlisting}[language=Python]
# Merge
def Merge(pre, post):
	t_list=[]
	i=0
	j=0

while i<len(pre) and j<len(post):
	if pre[i]<=post[j]:
		t_list.append(pre[i])
		i+=1
	else:
		t_list.append(post[j])
		j+=1

if i>=len(pre):
	t_list.extend(post[j:])
elif j>=len(post):
	t_list.extend(pre[i:])

return t_list
	\end{lstlisting}

	\section{과제 4}
	이진 트리를 생성하고 각 순회 결과 출력하기
	\subsection{클래스로 트리 구현}
	트리의 노드를 클래스로 구현하여 각 노드를 리스트에 추가 하였다. 
	 Node 클래스의 멤버변수는 key (key값), L(왼쪽 자식 노드), R(오른쪽 자식 노드)이 있다.
	\begin{lstlisting}[language=Python]
class Node:
	def __init__(self, Key, L, R):
		self.key=Key
		self.L=L
		self.R=R
	\end{lstlisting}
	\subsection{순회하기}
	 각 순회 함수를 Recursion으로 구현하여 결과를 출력하였다.
	PreOder 함수를 예시로 첨부
	\begin{lstlisting}[language=Python]
def PreOrder(now):
	print(my_tree[now].key)
	if my_tree[now].L!=-1:
		PreOrder(my_tree[now].L)
	if my_tree[now].R!=-1:
		PreOrder(my_tree[now].R)
	\end{lstlisting}

	\section{과제 5}
	강의실 배정하기
	\subsection{이중리스트로 입력 받기}
	각 강의의 번호와 시작, 종료 시간을 리스트로 입력 받은 후 다시 classes라는 리스트에 추가하였다.
	\begin{lstlisting}[language=Python]
# input
n=int(input())
classes=[]
for i in range(0, n, 1):
	my_class = list(map(int, input().split()))
	classes.append(my_class)
	\end{lstlisting}
	\subsection{시작 시간을 기준으로 정렬}
	itemgetter를 통해 classes를 각 강의의 시작 시간을 기준으로 정렬하였다.
	\begin{lstlisting}[language=Python]
from operator import itemgetter
classes.sort(key=itemgetter(1))
	\end{lstlisting}
	\subsection{강의실 배정}
	nowtime을 현재 들여다보는 강의의 시작으로 갱신해가며 강의 배정을 한다. playing에 있는 강의가 끝나면 result에 추가해준다.
	\begin{lstlisting}[language=Python]
now_time=0
playing=classes[0]
result=[]

for cl in classes:
	now_time=cl[1]

	# playing class is finished,
	if playing[2]<=now_time:
		result.append(playing[0])
		playing=cl
	# change playing class.
	elif playing[2]>cl[2]:
		playing=cl
	# append last class.
result.append(playing[0])
	\end{lstlisting}

	\section{과제 6}
	\subsection{입력 받기}
	str 리스트로 입력 받았다.
	\subsection{암호화 검사}
	test 함수에 검사할 code를 넣어준다. 암호화 되지 않은 경우 code를 '-'로 바꿔 주고, 암호화 된 경우 len(code)는 0이다.
	\begin{lstlisting}[language=Python]
def code_test(code):
	# 100
	if code[:3]=='100':
		code=code[3:]
	else:
		code='-'
		return code

	# ~
	while code[0]=='0':
		code=code[1:]

	# 1
	if code[0]=='1':
		code = code[1:]
	else:
		code='-'
		return code

	# ~|01
	# ~1
	if code[0]=='1':
		while code[0]=='1':
			code = code[1:]
		if code[0]=='0':
			code.insert(0,'1')
	# 01
	elif code[0]=='0':
		if code[:2]=='01':
			code = code[2:]
		else:
			code='-'
			return code
	else:
		code='-'
		return code

	return code
	\end{lstlisting}
	 
	
	
	
\end{document}