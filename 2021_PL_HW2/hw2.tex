% !TEX TS-program = pdflatex
% !TEX encoding = UTF-8 Unicode

\documentclass{article}

\usepackage{amsmath,amssymb}
\usepackage{kotex}
\usepackage{listings}
\usepackage{amssymb,mathtools}

\begin{document}
	\title{프로그래밍 언어 HW2}
	\author{B811181 조예린}
	\maketitle
	
	\section{과제 1}
	텍스트에서 'love'라는 단어가 몇 번 나오는지 카운트하여 출력하라.
	\subsection{정의절}
	정의절에서 love의 카운트 횟수를 저장할 int형 변수를 선언했다.
	\begin{lstlisting}
%{
#include <stdio.h>
int n_love = 0;
%}
	\end{lstlisting}
	\subsection{규칙절}
	1. love 또는 Love를 인식한 경우 변수의 값을 증가시켰다.
	\begin{lstlisting}
love|Love		{n_love++;}
	\end{lstlisting}
	2. love 또는 Love 가 아닌 문자나 줄바꿈 문자의 경우 아무것도 하지 않고 흘려주었다.
	\begin{lstlisting}
.|\n	;
	\end{lstlisting}

	\subsection{서브루틴절}
	love의 개수를 출력한다.
	\begin{lstlisting}
int main(){
	yylex();
	printf("number of love=%d\n", n_love);
	return 0;
}

int yywrap(){
	return 1;
}
	\end{lstlisting}



	\section{과제 2}
	(100$\sim$1$\sim|$01)$\sim$을 만나면 'is danger'을 출력하라.
	\subsection{규칙절 : 패턴}
	\begin{lstlisting}
(10(0+)1((1*)|0)1)+	{printf("%s is danger\n", yytext);}
	\end{lstlisting}
	규칙과 동일하게 패턴을 적용하였다. \\
	다만 $'1\sim|0'$ 부분의 구현을 살펴보면 무조건 앞에 1이 온 후 $(1*)|0$으로, 즉 1이 0번 이상 반복되거나 0이 무조건 한번 나와야하는 패턴을 짰다.\\
	연산 부분에서는 yytext로 읽어들인 문자 그대로 반환하여 'is danger'을 출력하였다.
	
	
	\section{과제 3}
	c코드를 읽어 해석한 것을 출력하라.
	\subsection{정의절}
	\subsubsection{} 각 패턴의 개수를 저장할 배열을 선언하였다.
	\begin{lstlisting}
%{
#include <stdio.h>
int n_arr[13] = { 0, };
%}
	\end{lstlisting}
	\subsubsection{} 규칙절에서 사용할 규칙을 정의하였다.
	\begin{lstlisting}
DIGIT	[0-9]
LETTER	[a-zA-Z]
NOTP	[a-oq-zA-OQ-Z]
	\end{lstlisting}

	\subsection{규칙절}
	lex에서는 먼저 작성한 패턴이 우선순위가 높으므로 과제설명 파일에 기재된 순서와 상관없이 우선순위가 높은 순서로 보고서를 작성하겠다.
	\\
	\subsubsection{comment} : 주석문의 개수 (n\_arr[5]에 저장)
	\\
	\\
	(1) 여러줄 주석 $(/*...*/)$
	\begin{lstlisting}
"/*"(.*|\n*)*"*/"	{n_arr[5]++;}
	\end{lstlisting}
	우선 $/*$와 $*/$는 문자 그대로 해석하기 위해 큰 따옴표("")로 감싸주었다.
	\\
	$"/*"$와 $"*/"$로 감싸진 중간 패턴을 살펴보자.\\
	주석 안에 있는 어떤 코드든 주석이 끝날 때까지 무의미하게 흘려보내야하기 때문에 '.'과 줄바꿈 문자를 사용하였다. 어떤 문자나 줄바꿈 문자가 있든 없든, 또는 둘 중 어떤 것이 오거나 반복되는지에 상관없이 다 인식할 수 있도록 이와 같은 패턴으로 작성하였다. 
	\\
	\\
	(2) 한줄 주석 (//...)
	\begin{lstlisting}
"//"(.)*\n	{n_arr[5]++;}
	\end{lstlisting}
	한줄 주석은 //으로 시작한 후 줄바꿈 문자가 나오면 끝나므로 위와 같이 작성하였다.
	\\
	 \subsubsection{Preprocessor} : $\sharp$include, $\sharp$define 전처리문의 개수 (n\_arr[0]에 저장)
	\begin{lstlisting}
(#include|#define)(.*)\n	{n_arr[0]++;}
	\end{lstlisting}
	$\sharp$include 혹은 $\sharp$define이 나오면 그 한 줄을 전처리문으로 인식한다.
	\\
	\subsubsection{octal number} : 8진법 숫자 개수 (n\_arr[1]에 저장)
	\begin{lstlisting}
0{DIGIT}+		{n_arr[1]++;}
	\end{lstlisting}
	c언어에서 8진수는 0으로 시작하는 숫자이기 때문에 앞에 0이 나오고 그 뒤에 숫자가 1번 이상 나와야하는 패턴으로 작성하였다. (0이 한번만 나온다면 숫자 0을 의미하므로 1번 이상 반복으로 해주었다.) \\
	이 때 정의절에서 미리 정의해둔 DIGIT을 사용하였다.
	\\
	\subsubsection{negative decimal number} : 10진법 숫자 중 음수의 개수 (n\_arr[2]에 저장)
	\begin{lstlisting}
-{DIGIT}+	{n_arr[2]++;}
	\end{lstlisting}
	십진법 음수는 '-'로 시작하고 숫자가 한개 이상 있어야하므로 이와 같은 패턴으로 구현하였다.
	\\
	\subsubsection{positive decimal number} : 10진법 숫자 중 양수의 개수 (n\_arr[3]에 저장)
	\begin{lstlisting}
{DIGIT}+	{n_arr[3]++;}
	\end{lstlisting}
	0으로 시작하는 숫자, 즉 8진법은 위에서 인식됐으므로 이와 같이 코드를 작성해도 가능하다.
	\\
	\subsubsection{operator} : 연산자의 개수 (n\_arr[4]에 저장)
	\begin{lstlisting}
"+"|"-"|"*"|"/"|"%"	{n_arr[4]++;}
==|!=|>|<|>=|<=		{n_arr[4]++;}
&&|"||"|!		{n_arr[4]++;}
"++"|"--"		{n_arr[4]++;}
","|"&"|"*"|"->"	{n_arr[4]++;}
	\end{lstlisting}

	\subsubsection{'=' '$\{$' '$\}$'} 
	: 각 기호의 개수 (n\_arr[6],[7],[8]에 각각 저장)
	\begin{lstlisting}
=	{n_arr[6]++;}
"{"	{n_arr[7]++;}
"}"	{n_arr[8]++;}
	\end{lstlisting}

	\subsubsection{wordcase}
	(1) wordcase1 : p가 두 번만 들어간 단어의 개수 (n\_arr[9]에 저장) 
	 \begin{lstlisting}
{NOTP}*p{NOTP}*p{NOTP}*	{n_arr[9]++;}
	\end{lstlisting}
	정의절에서 미리 구현해둔 NOTP를 사용하였다.\\
	p가 정확히 두 번 존재해야하기 때문에 pp의 앞, 중간 사이, 끝에 {NOTP}*를 넣어주었다. 
	\\
	\\
	(2) wordcase2 : e로 시작하고 마지막 글자가 m인 단어의 개수 (n\_arr[10]에 저장) 
	\begin{lstlisting}
e{LETTER}*m	{n_arr[10]++;}
	\end{lstlisting}
	정의절에서 미리 구현해둔 LETTER를 사용하였다.
	\\
	\subsubsection{word}
	: 그 외 단어의 개수 (n\_arr[11]에 저장) 
	\begin{lstlisting}
{LETTER}+	{n_arr[11]++;}
	\end{lstlisting}
	
	\subsubsection{mark}
	: 위에서 count 되지 않은 문자의 개수 (n\_arr[12]에 저장) 
	\begin{lstlisting}
.|\n	{n_arr[12]++;}	
	\end{lstlisting}	

	\subsection{서브루틴절}
 	정해진 형식에 맞춰 결과를 출력한다.
	\begin{lstlisting}
int main(){
	yylex();
	printf("preprocessor = %d\n", n_arr[0]);
	printf("octal number = %d\n", n_arr[1]);
	printf("negative decimal number = %d\n", n_arr[2]);
	printf("positive decimal number = %d\n", n_arr[3]);
	printf("operator = %d\n", n_arr[4]);
	printf("comment = %d\n", n_arr[5]);
	printf("'=' = %d\n", n_arr[6]);
	printf("'{' = %d\n", n_arr[7]);
		printf("'}' = %d\n", n_arr[8]);
	printf("wordcase1 = %d\n", n_arr[9]);
	printf("wordcase2 = %d\n", n_arr[10]);
	printf("word = %d\n", n_arr[11]);
	printf("mark = %d\n", n_arr[12]);
	return 0;
}

int yywrap(){
	return 1;
}
	\end{lstlisting}
	


\end{document}